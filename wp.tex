\section{The Word Problem}
\label{sec:wp}
We now turn our attention to the second important result to do with free groups and presentations. In this section our goal is to investigate to what extent computers can understand group presentations. In particular we will ask whether or not a ``Turing machine'', which we shall define, can check whether a group given by its presentation is in fact trivial. To give some more intuition to what we plan to show, we will begin by talking about decision processes.

If $\mathcal{L}$ is a collection of ``questions'', we say that a \emph{decision process} or \emph{algorithm} is a deterministic process, or set of directions that are applied to questions
in $\mathcal{L}$ and always give us an answer after a finite number of steps. This is best understood with an example problem, and no better example exists for our purposes than the word problem.

Let $G$ be a group that is finitely generated and we suppose that we are given the presentation of this group which is
\begin{equation*}
  G = (x_1,x_2\dots,x_n \mid r_j,j \geq 1).
\end{equation*}
Thus every word that may or not be reduced determines a not necessarily unique element in $G$ and every element and $G$ can be determined by such a word as well. This is from Corollary \ref{cor:presentations-exist} which gives us that if $R$ is the normal subgroup generated by $\{r_j \mid j \geq 1\}$, elements of $G$ have the form $\omega R$ where $\omega$ is a word on $\{x_1,x_2 \dots x_n\}$. Note that this is not yet in full generality since we limit our discussion to finitely generated groups here. Let $\mathcal{L}$ be the set of questions of the form: ``If $\omega$ is a word on $\{x_1,x_2, \dots x_n \}$ is $\omega = 1$ in $G$?''.
If there exists a decision process that can answer each question in $\mathcal{L}$ we say that $G$ has a \emph{solvable word problem}.

The answer to this is in general not straightforward, however we have will illustrate a basic solution to the word problem for the free group in particular. For a finite number $n$ of generators we can index all the not necessarily reduced words that we encounter as follows. Let $\omega_0$ determine the empty word that is equal to $1$ in $G$. Next consider words of length $1$ and order them via $x_i < x_i^{-1}< x_{i+1}$ for all $i$. This would give us the words $\omega_1$ to $\omega_{2n}$. Beyond this we can repeat this formula ordering words of length $n$ greater than $1$ based on lexicographic order. For example words of length $2$ would be ordered via
\begin{equation*}
  x_1x_1 < x_1x_1^{-1} < x_1x_2 < \dots < x_1^{-1}x_1 < x_1^{-1}x_1^{-1} < \dots < x_n^{-1}x_n^{-1}
\end{equation*}
to give us the words $\omega_{2n + 1}$ to $\omega_{2n + (2n)^2}$. In this manner we can proceed by saying words of length $n$ are ``greater'' than words of length $n-1$ and index all our words.

With this ordering, we let $\mathcal{L}$ be questions of the form: ``Is $w_k = 1$ in G?''. In the case where $G$ is free, i.e.\ has the presentation
\begin{equation*}
  (x_1,x_2 \dots x_n \mid ),
\end{equation*}
questions in $\mathcal{L}$ can be answered by simply reducing the word and checking whether it reduces to $1$. Here is a decision process that expresses this idea.

\begin{enumerate}
\item If the length of the word $w$ is $0$ or $1$ proceed to step 3. If not underline the first adjacent pair of letters, if any, of the form $x_ix_i^{-1}$ or $x_i^{-1}x_i$. If there are no two such letters, underline the last two letters of $w$. Proceed to step 2.
\item If the underlined letters are of the form $x_ix_i^{-1}$ or $x_i^{-1}x_i$, delete them in the word and return to step 1; if not proceed to step 3.
\item If the word is empty conclude that $w = 1$ otherwise we have $w \neq 1$.
\end{enumerate}

To solve the word problem in greater generality or even to investigate  whether the problem is solvable, we will need to develop a more formal understanding of the concepts that we have described in this section.

\section{Turing Machines and Decision Processes}

The primary object that we wish to study is the \emph{Turing Machine}, which abstracts the notion of decision processes in a manner that we can think about mathematically. We imagine the Turing machine as a box with a tape that running through it. The tape is divided into cells that run along it, with each cell containing one of the finitely many symbols $s_0,s_1,\dots,s_M$. For the sake of generality we assume that the tape is of infinite length i.e.\ contains infinitely many cells with symbols on them. The box, we assume, is at any point in time ``scanning'' a particular cell on the tape and has a ``state'', one of $q_0,q_1,\dots q_N$. From this moment the machine can complete one of the following ``instruction'', depending on the current symbol being scanned and current state.
\begin{enumerate}
\item Replace the symbol $s_j$ by the symbol $s_k$ and continue scanning the same cell. Go into state $q_l$.
\item Move the tape to scan the cell to the right. Go into state $q_l$.
\item Move the tape to scan the cell to the left. Go into state $q_l$.
\end{enumerate}
We start by setting the machine to a ``starting state'' $q_1$ and set it to  scan a particular cell on tape we give to it which has any symbol printed on it. Now, we allow the machine to run which we understand to mean: execute the next instruction. Keep in mind since the state the machine is in and symbol it is scanning determine the unique next instruction, there is no ambiguity at each instruction as to what the next instruction is. We understand the machine to ``stop'' if it enters state $q_0$ after completing finitely many instructions, though it is not guaranteed that the machine must indeed stop. Now with this intuition, the formal definitions that follow should appear natural.

\begin{definition}
  We say that a \emph{quadruple} is a $4$-tuple of symbols that can be written in one of the following ways.
  \begin{align*}
    & q_is_js_kq_1, \\
    & q_is_jRq_l, \\
    & q_is_jLq_l.
  \end{align*}
  A \emph{Turing machine}, $T$ is a finite set of of quadruples where no two quadruples have the same first two symbols. The set of all middle two symbols in elements of $T$ ($s$-letters) form the \emph{alphabet} $\{s_0, \dots, s_1\}$ of the Turing machine. If the alphabet of a Turing machine can be written as the disjoint union $S \cup T$, then we say an \emph{s-word} on $S$ is a positive word on $S$.
\end{definition}

> blank letters

We can see that this definition is a more concrete form of the description of Turing machines that we have given above. The \emph{alphabet} form the finitely many symbols that are printed on the tape and the quadruples tell us without ambiguity which of the ``instructions'' from above that we must perform. The quadruple of the first type can be understood to mean: ``If we are in state $q_i$ whilst scanning a cell with symbol $s_j$, then print the symbol $s_k$ on that cell and change the machine's state to $q_l$''. The second and third types would mean: ``If we are in state $q_i$ and scanning a cell with symbol $s_j$, then move either left/right and enter state $q_l$''.

In our definition of the Turing Machine, all we define is: what symbols may exist on the tape, what states may the machine enter and what instructions can be performed. It describes what the ``box'', in the informal exposition above, can do. To study how a Turing machine $T$ runs, we need a way of talking about how it is started, i.e.\ in a particular state $q_i$ and scanning a particular cell of a provided tape, but this information is what is all that we can see in our notional box and tape at any point during the running of the machine. For this we have the instantaneous description, defined below, which tells us what symbols are present on the cells of the tape, which cell is being scanned and what state the machine is in.

\begin{definition}
  An \emph{instantaneous description} is the positive word $\alpha$ of the form $\sigma q_i \tau$, where $\sigma$ and $\tau$ are s-words and $\tau$ is nonempty.
\end{definition}

What we understand from this is that the tape is composed of cells whose symbols from left to right form the word $\sigma\tau$. The machine is in state $q_i$ and scanning the cell represented by the first letter of $\tau$.

\begin{definition}
  Let $T$ be a Turing machine. We say that a \emph{basic move} is an ordered pair of instantaneous descriptions $(\alpha,\beta)$, that we denote by
  \begin{equation*}
    \alpha \mapsto \beta,
  \end{equation*}
  and for which there must exist s-words $\sigma$ and $\sigma'$ that may be empty, such that one of the following conditions hold:
  \begin{enumerate}
  \item $\alpha = \sigma q_i s_j\sigma'$, $\beta = \sigma q_l s_k \sigma'$ and we must have the quadruple $q_is_js_kq_l$ in $T$.
  \item $\alpha = \sigma q_i s_j s_k \sigma'$, $\beta = \sigma s_j q_l s_k \sigma'$ and we must have the quadruple $q_is_jRq_l$ in $T$.
  \item $\alpha = \sigma q_i s_j$, $\beta = \sigma s_j q_l s_0$ and we must have the quadruple $q_is_jRq_l$ in $T$.
  \item $\alpha = \sigma s_k q_i s_j \sigma'$, $\beta = \sigma q_l s_k s_j \sigma'$ and we must have the quadruple $q_is_jLq_l$ in $T$.
  \item $\alpha = q_i s_j \sigma'$, $\beta =  q_l s_0 s_j \sigma'$ and we must have the quadruple $q_is_jLq_l$ in $T$.
  \end{enumerate}
\end{definition}

Basic moves are those pairs of instantaneous descriptions by which it is possible to go from the first to the second by the Turing machine performing one of its instructions. The requirement in the definition of Turing machines that no two quadruples are to have the same first two letters give us that there is no ambiguity in what instruction to follow i.e.\ for an instantaneous description $\alpha$, there is a unique instantaneous description $\beta$ such that $\alpha \mapsto \beta$. The reason we have five conditions in the above definition despite having only three types of instructions is that we need two extra conditions for what occurs when we reach the ``end'' of the tape on the left and right side. As implied by the definition, we simply pretend that all the cells beyond the description contain the blank symbol $s_0$.

\begin{definition}
For a Turing machine $T$, an instantaneous description $\alpha$ is \emph{terminal} if there is no instantaneous description $\beta$ such that $\alpha \mapsto \beta$, i.e.\ that $(\alpha,\beta)$ is a basic move. We say that $T$ \emph{computes} $\omega$, a positive word on the alphabet of $T$ if for instantaneous description $\alpha_1 = q_1\omega$, there exists a finite number of instantaneous descriptions $\alpha_2,\dots \alpha_t$ where for all $i$ positive, $\alpha_i \mapsto \alpha_{i+1}$ and $\alpha_t$ is terminal.
\end{definition}

Informally, \emph{computing} the word $\omega$ entails printing the characters o $\omega$ on the cells of the tape from left to right, setting the Turing machine into state $q_1$ and having it scan the first character of the tape and having it run till it stops, which occurs when there are no more possible moves to perform. The Turing machine may however run forever, in which case we don't say that it computes the word $\omega$.

\begin{definition}
  If $\Omega$ is the set of all positive words on the set of symbols $S = \{s_1, s_2 \dots, s_M\}$, we define for any Turing machine $T$, the subset $e(T)$ of $\Omega$ by
  \begin{equation*}
    e(T) = \{\omega \in \Omega \mid T \text{ computes } \omega\}.
  \end{equation*}
  We say that $e(T)$ is the set \emph{enumerated by $T$}. We say that a subset $E \subseteq \Omega$ is \emph{recursively enumerable} if there exists a Turing machine $T$ that enumerates $E$; that is, such that  $e(T) = E$.
\end{definition}

For every Turing machine $T$, with alphabet $S$, there must be some words of $S$ that are computed by $T$ and some that aren't and so the subset $e(T) \subseteq \Omega$ is well defined. Actually finding which words are in it though, with our current background, can only be done by running the Turing machine $T$ with the word in question on the tape. But checking whether or not the word is computed entails by definition waiting forever! What we now want to investigate is which subsets $E$ of $\Omega$ do we have to wait infinitely long for to check whether a positive word in $E$ lies in $e(T)$. If $E$ and $\Omega - E$ are recursively enumerable, then there exist Turing machine $T$ and $T'$ such that $E = e(T)$ and $(\Omega - E) = e(T')$. Thus setting up $T$ and $T'$ to simultaneously compute any word $\omega \in \Omega$, one of the machines will terminate in finite time, indicating to us whether or not $\omega$ lies in $E$. We define this condition as follows.

\begin{definition}
  Let $\Omega$ be the set of all positive words on $S = \{s_0,s_1, \dots, s_M \}$. We say that a subset $E \subseteq \Omega$ is \emph{recursive} if both $E$ and its complement $\Omega - E$ are recursively enumerable.
\end{definition}

Now with this formal background at our disposal, we can more concretely present the word problem that we introduced informally at the beginning of this section.

\begin{definition}
  \label{def:wp}
  Let $G$ is a group with the presentation $(x_1,x_2, \dots, x_n \mid \Delta)$, and $\Omega$ is the set of all words on $\{x_1, \dots x_n\}$ (which we consider positive words on $X \cup X^{-1}$). Then, we say that $G$ has a \emph{solvable word problem} if the set
  \begin{equation*}
    \{\omega \in \Omega \mid \omega = 1 \text{ in } G \}
  \end{equation*}
  is recursive.
\end{definition}

If a group has a solvable word problem, it implies that for any word $\omega$ on the generators, there exist Turing machines $T$ and $T'$ that can be run simultaneously to tell us whether or not $\omega$ evaluates to $1$ in $G$, which is essentially identical to the intuitive form of problem that we presented in Section \ref{sec:wp}.

In Section \ref{sec:wp} we also presented a way to associate words on a finite set of symbols with the natural numbers (lexicographically), in the case that our symbols are countable and can be enumerated, it is useful to define the following.
\begin{definition}
  If $s_1$ is a set of finite symbols and the words on $s_1$ can be indexed by $s_1^{n+1}$, then we say that $E$ is a recursively enumerable subset of $\mathbb{N}$ if there exists a Turing machine $T$ that has the symbols of $s_1$ in its alphabet and
  \begin{equation*}
    E = \{n \in \mathbb{N} \mid T \text{ computes } s_1^{n+1}\}.
  \end{equation*}
\end{definition}

We then have the following theorem which will prove important when we investigate the word problem for groups.

\begin{lemma}
  \label{lem:turing-countable}
  There exist countably many Turing machines.
\end{lemma}

\begin{proof}
  The definition gives us that a Turing machine $T$ is a finite set of quadruples. We can assign a unique natural number to each Turing machine in the following manner; we start by assigning numbers to the symbols in the quadruples as follows:
  \begin{align*}
    R &\mapsto 0; \\
    L &\mapsto 1; \\
    q_0 \mapsto 2; \quad q_1 &\mapsto 4; \quad q_2 \mapsto 6; \dots \\
    s_0 \mapsto 3; \quad s_1 &\mapsto 5; \quad s_2 \mapsto 7; \dots \\
  \end{align*}
  and it should be easy to see that each possible symbol of the Turing machine has a corresponding unique integer $e$ assigned to it. For Turing machine $T$, let $w(T)$ be the not necessarily unique juxtaposition of the quadruples that would then have length $4m$ for some $m \in \mathbb{N}$. It is easy to see that the word $w(T)$ is unique for the Turing machine $T$. If $p_i$ is the $i$'th prime number and $e_i$ is the natural number assigned to the $i$'th symbol in $w(T)$, we define the \emph{G\"odel number} by
  \begin{equation*}
    G(T) = \prod_{i=1}^{4m} p_i^{e_i}.
  \end{equation*}
  By the unique prime factorisation of natural numbers, it is clear that each Turing machine has a unique G\"odel number though since our choice of juxtaposing the quadruples was arbitrary, a G\"odel number doesn't correspond to a unique Turing machine. However this is enough to prove our claim that all the set of all Turing machines that may exist are in fact countable.
\end{proof}

\begin{theorem}
  \label{thm:natural}
  There exists a recursively enumerable subset of $\mathbb{N}$ that is not recursive.
\end{theorem}

\begin{proof}
  To prove our Theorem, we index Turing machines by $T_1,T_2, \dots$ which we can do by Lemma and consider the set
  \begin{equation*}
    E = \{ n \in \mathbb{N} \mid T_n \text{ computes }s_1^{n+1} \}.
  \end{equation*}
  We will show that $E$ is recursively enumerable but not recursive. Since Turing machine $T_n$, computes $s_1^{n+1}$, the following sequence of basic moves terminates
  \begin{equation*}
    q_1 s_1^{n+1} \mapsto \alpha_{n+1,2} \mapsto \alpha_{n+1,3} \mapsto \dots \mapsto \alpha_{n+1,m}
  \end{equation*}
  With this, we can construct a Turing machine $T^*$ that ``traverses'' the following diagram along the arrows. This traversal may not be continuous depending on the number of basic moves a $T_i$ takes to compute a  $q_1 s_1^i$. In fact $T^*$ can be defined such that these terminal instantaneous descriptions $\alpha_n$ contain the stopping state $q_0$.
    \begin{equation*}
    \begin{tikzcd}
      T_0 & T_1 & T_2 & T_3 \\
      q_1 s_1^1 \arrow[d] & q_1 s_1^2 \arrow[r] & q_1 s_1^3 \arrow[dl] & q_1 s_1^4\arrow[r] & \dots \\
      \alpha_{1,2} \arrow[ur] & \alpha_{2,2} \arrow[dl]  & \alpha_{3,2} \arrow[ur]. \\
      \alpha_{1,3} \arrow[d]& \alpha_{2,3} \arrow[ur] \\
      \alpha_{1,4} \arrow[ur]\\
    \end{tikzcd}
  \end{equation*}

It should be easy to see that each $T^*$ computes each natural number by the traversal in the above and so $E$ is enumerated by $T^*$.

Now we will show that $E$ is not recursive, i.e.\ that $\mathbb{N} - E$ is not enumerable. To see this we suppose to the contrary that $T'$ enumerates $\mathbb{N} - E$, but $T' = T_m$ already for we have already indexed all Turing machines and so $m \not\in \mathbb{N} - E$ is computed by $T'$, a contradiction. It follows that $E$ is recursively enumerable but not recursive.
\end{proof}

In this section we have formally introduced Turing machines; but this begs the question? Does this formal definition coincide with our informal discussion at the start of this chapter involving decision processes? The reader should recall that a decision process is a list of questions $\mathcal{L}$ such that a deterministic process leads to an answer, for one of the questions in $\mathcal{L}$. Indeed the Turing machine is deterministic in that running the Turing machine requires no external control since at each point in ``time'' the quadruples uniquely determine the machine's next move. The question then is to what extent being able to answer questions $\mathcal{L}$ coincide with a set being recursive. This is not a question that has a definite answer since the notion of decision processes is necessarily intuitive and not one of formal mathematics.

In an attempt to contribute to questions regarding the foundation of mathematics at that time, in 1932 Alonzo Church published a paper introducing a system that tried to do away with some of the problems with formal logic at the time \cite{church1932}. Though it was later found that this system was flawed with an inconsistency \cite{kleene-rosser1935}, Church refined his system and introduced \emph{Lambda Calculus} in a 1936 paper \cite{church1936}. This system would prove a model equivalent to the Turing machine and both these formal systems have been posited to be the ``formal'' description of decision processes and would lead to the study of Computability Theory. This position, of understanding decision processes in this way is called the \emph{Church-Turing Thesis}.

\section{The Markov-Post Theorem}
The question at hand is whether or not a group, given its presentation, has a solvable word problem. First, we will study this question in the case of finitely presented semigroups which we shall introduce in section \ref{sec:mpt:semi}. As it turns out, there does exist a semigroup for which the word problem on it, which we shall presently define, is unsolvable. This was proved by Emil Post in 1947 \cite{post1947} and shortly after by Andrey Markov \cite{markov1951}. We will need the Markov Post Theorem in order to prove the insolubility of the word problem for some group and its presentation. 

\subsection{Semigroups}
\label{sec:mpt:semi}

\begin{definition}
  A semigroup $(G,*)$, is a set $G$ equipped with a binary operation $*$, that is associative.
\end{definition}

Semigroups are more general than groups, not necessarily requiring an identity element and thereby no condition on inverses either. As with groups we may omit specifying the symbol representing the binary operation and denote the ``multiplication'' as we do in groups.
Recall that homomorphisms on groups are well defined without any reference to the identity element of the group. Thus we can use the same definition on semigroups as well. This is all we will need to define \emph{free semigroups}.

\begin{definition}
  If $X$ is a subset of a  of a semigroup $\Sigma$, and for every semigroup $S$ and function $f:X \rightarrow S$ there exists a unique homomorphism $\phi: \Sigma \rightarrow S$ that extends $f$, then we say that $\Sigma$ is a \emph{free semigroup}.
\end{definition}

For the sets and functions in the above definition, the following diagram commutes:
\begin{equation}
  \begin{tikzcd}[ampersand replacement=\&, sep=huge]
    \Sigma \arrow[dr,dashrightarrow,"\phi"] \\
    X \arrow[u,hook] \arrow[r,"f"] \& S.
  \end{tikzcd}
\end{equation}

\begin{definition}
  A \emph{congruence} on a semigroup $S$ is an equivalence relation denoted by $\equiv$ on $S$ such that
  \begin{equation*}
    a \equiv a' \quad\text{and}\quad b \equiv b' \qquad\text{imply that}\qquad ab \equiv a'b'.
  \end{equation*}
  The \emph{quotient semigroup} for a congruence $\equiv$ is the set of all equivalence classes, denoted by $S/\equiv$, along with the operation
  \begin{equation*}
    [a][b] := [ab].
  \end{equation*}
\end{definition}

An easy example of a semigroup congruence is induced by a homomorphism between semigroups $\phi:S \rightarrow T$.

\begin{example}
  If $S,T$ are semigroups and $\phi:S \rightarrow T$ is a homomorphism, there is a natural congruence on $S$, defined by $\alpha \equiv \beta$ if and only if $\phi(\alpha) = \phi(\beta)$ in $T$. By definition, if $\alpha' \equiv \beta'$, we have
  \begin{equation*}
    \phi(\alpha\beta) = \phi(\alpha)\phi(\beta) = \phi(\alpha')\phi(\beta') = \phi(\alpha'\beta'),
  \end{equation*}
  and hence $\alpha\beta \equiv \alpha'\beta'$ as required. If we call this congruence $\ker \phi$, we have that $S/\ker \phi \cong \phi(S)$ as a semigroup. This is easy to see after noting that $[a] \in S/\ker \phi$ consists of all elements $x \in S$ such that $\phi(x) = \phi(a)$.
\end{example}

Since a relation $\sim$ on a set $S$ is nothing but a subset on $S \times S$ (where an element $(a,b)$ belongs to the subset if and only if $a \sim b$), we have this for congruences as well. For any subset $E \subseteq S \times S$, we can define the congruence generated by $E$ as the smallest congruences that has $E$ as it's subset. This is well defined because the whole $S \times S$, considered as a subset of itself, is a congruence. With this we may define the following.

\begin{definition}
  Let $\Sigma$ be a free semigroup with basis $X$ and suppose $\{w_i = u_i \mid i \in I \}$ is a family of equations where $w_i,u_i \in \Sigma$. If $\equiv$ be the congruence generated by the subset $\{(w_i,u_i) \mid i \in I \}$ of $\Sigma \times \Sigma$, then we say that the \emph{presentation} of the quotient semigroup $\Sigma/\equiv$ is:
  \begin{equation*}
    (X \mid w_i = u_i \text{ for all } i \in I).
  \end{equation*}
\end{definition}

\subsection{The Word Problem for Semigroups}
\label{sec:mpt:wps}

In this subsection, we will work towards the Markov Post Theorem. which will show us that there is a semigroup that has an unsolvable word problem, in a similar way to what we introduced formally in Definition \ref{def:wp}. Here is a precise definition of what we want to show for some semigroup.

\begin{definition}
  Let $\Gamma$ be a semigroup with generators $X = \{x_1, \dots, x_n\}$ and if $\Omega$ is the set of all positive words on $X$, we say that $\Gamma$ has an unsolvable word problem if there is a word $\omega_0 \in \Omega$ such that $\{\omega \in \Omega \mid \omega = \omega_0 \text{ in } \Gamma\}$ is not recursive.
\end{definition}

\begin{theorem}
  \label{thm:semi-re}
  For a finitely presented group with the following presentation
  \begin{equation*}
    G = (x_1, x_2, \dots x_n \mid r_1, \dots, r_m), 
  \end{equation*}
  and the set $\Omega$ comprised of all words on $\{x_1, \dots, x_n \}$, we have that the set $E = \{\omega \in \Omega \mid \omega = 1 \text{ in } G\}$ is recursively enumerable.
\end{theorem}

\begin{proof}
  We can list the words in $\Omega$ in Lexographic order and consequently index them against non-negative integers as we did towards the end of Section \ref{sec:wp}. I.e.\ we can write the elements of $\Omega$ as $\omega_0, \omega_1, \omega_1, \dots$.
  
  Now suppose for $k \in \{1,2,\dots,n\}$ that $r_{m+k}$ is the word $x_k x_k^{-1}$ . Then, as before, we can lexographically index all words on $\{ r_1, \dots, r_{m+n}\}$ by the natural numbers and write them as $\rho_1,\rho_2, \dots$ for all the countably many words. With this, we have that
  \begin{equation*}
    E = \{\omega_k\rho_l\omega_k^{-1} \mid k,l \in \mathbb{N}_0\}.
  \end{equation*}

  Thus we can order all the countably many elements of $E$ ensuring that, we eventually hit all of them as in \ref{eq:enumerate-e}.
  \begin{equation}
    \label{eq:enumerate-e}
    \begin{tikzcd}
%%      T_0 & T_1 & T_2 & T_3 \\
      \omega_0\rho_0\omega_0^{-1} \arrow[d] & \omega_0\rho_1\omega_0^{-1} \arrow[r] & \omega_0\rho_2\omega_0^{-1} \arrow[dl] & \omega_0\rho_3\omega_0^{-1}\arrow[r] & \dots \\
      \omega_1\rho_0\omega_1^{-1} \arrow[ur] & \omega_1\rho_1\omega_1^{-1} \arrow[dl]  & \omega_1\rho_2\omega_1^{-1} \arrow[ur]. \\
      \omega_2\rho_0\omega_2^{-1} \arrow[d]& \omega_2\rho_1\omega_2^{-1} \arrow[ur] \\
      \omega_3\rho_0\omega_3^{-1} \arrow[ur]\\
    \end{tikzcd}
  \end{equation}
  Actually, this is enough to tell us that there exists a Turing machine $T^*$ that computes $E$ by Theorem \ref{thm:natural}. 
\end{proof}

Thus our question investigating whether the set of words $E$ on the generators of $G$ is recursive, is half solved; we know that $E$ is recursively enumerable and it remains to ask whether $E$'s complement is recursively enumerable as well.

It might have occurred to the reader so far, that the descriptions of the Turing Machine, naturally feel like semigroups in that basic moves represent a ``directed'' relation on instantaneous descriptions; if $\alpha \mapsto \beta$, then considering $\alpha,\beta$ as elements of a semigroup that has generators containing symbols of the Turing machine, $\alpha = \beta$ feels like a natural relation except for the fact that $\beta \mapsto \alpha$ is not necessarily a basic move. With this in mind we define the following. Note here that we introduce extra symbols $q$ and $h$. $h$ can be understood as a marker that maps the end of the tape and we will soon see what the importance of $q$ is.

\begin{definition}
  \label{associated-semigroup}
  Let $T$ be a Turing machine that has stopping state quadruples with $s$-letters $\{s_0,s_1,\dots,s_M\}$ and $q$-letters $\{q_0,q_2,\dots,q_N\}$ where $q_0$ is understood to be the Turing machine's stopping state. If $h$ and $q$ are two more symbols, we define the $\Gamma(T)$, the \emph{associated semigroup} of $T$ to be the semigroup with presentation
  \begin{equation*}
    \Gamma(T) = (q,h,s_0,s_1,s_2, \dots, s_M, q_0,q_2, \dots, q_N \mid R(T))
  \end{equation*}
  where we will $R(T)$ is the set of relations with the following elements:
  \begin{itemize}
  \item  $q_is_j = q_ls_k$ if  $q_is_js_kq_l$ is a quadruple in $T$;
  \item $q_is_js_\beta = s_jq_ls_\beta$ if $q_is_jRq_l$ is a quadruple in $T$ and $\beta \in \{1,2,\dots M\}$;
  \item $q_is_jh = s_jq_ls_0h$ if $q_is_jRq_l$ is a quadruple in $T$;
  \item $s_\beta s_jq_i = q_ls_\beta s_j$ if $q_is_jLq_l$ is a quadruple in $T$ and $\beta \in \{1,2,\dots M\}$;
  \item $hq_is_j = h q_ls_0s_j$ if $q_is_jRq_l$ is a quadruple in $T$;
  \item $q_0s_\beta = q_0$ if $\beta \in \{1,2,\dots M\}$;
  \item $s_\beta q_0 h = q_0h$ if $\beta \in \{1,2,\dots M\}$;
  \item $hq_0h = q$.
  \end{itemize}
\end{definition}

For the rest of the project we make the following definition of congruence on semigroups. Let $\omega$ and $\omega'$ be any two words on an alphabet $X$. We say that 
\begin{equation*}
  \omega \equiv \omega'
\end{equation*}
if and only if the two words have the exact same spelling. Furthermore if $\Gamma$ is a semigroup with presentation $(X \mid \alpha_i = \beta_j,j \in J )$, we have that $\omega = \omega'$ exactly when there is a finite sequence of \emph{elementary operations}
\begin{equation*}
  \omega \equiv \omega_1 \rightarrow \omega_2 \rightarrow \dots \rightarrow \omega_t \equiv \omega',
\end{equation*}
and the elementary operation $\omega_i \rightarrow \omega_{i+1}$ must have that $\omega_i \equiv \sigma\alpha_j\tau$ and $\omega_{i+1} \equiv \sigma\beta_j\tau$ for some subwords $\sigma,\tau$.

\begin{definition}
  A word in the associated semigroup is \emph{$h$-special} if it has the spelling $h\alpha h$ for some  instantaneous description $\alpha$.
\end{definition}


\begin{lemma}
  \label{lem:gamma-prop}
  If $T$ is a Turing machine that has stopping state $q_0$ and associated semigroup $\Gamma(T)$, with presentation
  \begin{equation*}
    \Gamma(T) = (q,h,s_0,s_1,s_2, \dots, s_M, q_0,q_2, \dots, q_N \mid R(T)), 
  \end{equation*}
  we have the following
  \begin{enumerate}
  \item Let $\omega$ and $\omega'$ be words on $\{q,h,s_0,s_1,s_2, \dots, s_M, q_0,q_2\}$ where neither $\omega$ nor $\omega'$ have the spelling $q$. Then if $\omega \rightarrow \omega'$ is an elementary operation then $\omega$ is $h$-special if and only if $\omega'$ is.
  \item If $\omega = h\alpha h$ is $h$-special, $\omega'$ doesn't spell $q$ and $\omega \rightarrow \omega'$ is an elementary operation of one of the first five types of relations in Definition \ref{associated-semigroup}, then $\omega' \equiv h\beta h$  where either $\alpha \mapsto \beta$ or $\beta \mapsto \alpha$ is a basic move of $T$.
  \end{enumerate}
\end{lemma}

\begin{proof}
  \begin{enumerate}
  \item The only relation that creates or destroys $h$ is the last relation listed in Definition \ref{associated-semigroup}. Consequently, since we don't allow either word to spell $q$, the elementary move can't be the last one and the desired result follows.
  \item We know from the above that $\omega'$ must be $h$-special and we assume it can be written $h\beta h$ for some instantaneous description $\beta$. The elementary operation $\omega \rightarrow \omega'$ substitutes a subword of $\omega$ with based on the one of the first five relations in Definition \ref{associated-semigroup}. This elementary move corresponds to a quadruple of the Turing machine. Since the elementary operation can occur in either direction but the basic moves governed by the quadruples only move in one direction we either have $\alpha \mapsto \beta$ \emph{or} $\beta \mapsto \alpha$.
  \end{enumerate}
\end{proof}

\begin{lemma}
  \label{lem:semi-turing}
  Let $T$ be a Turing machine with stopping state $q_0$, let $\Omega$ be the set of all positive words on the alphabet of $T$, and let $E = e(T)$. If $\omega \in \Omega$, then $\omega \in E$ if and only if $hq_1\omega h = q$ in the associated semigroup $\Gamma(T)$.
\end{lemma}

\begin{proof}
  Suppose $\omega \in E$, or equivalently $T$ computes $\omega$. Thus we must have a finite sequence of basic moves in $T$ that take $q_1\omega$ to a terminal state. In other words there exist instantaneous descriptions
  \begin{equation}
    \label{eq:bmoves}
    q_1\omega \equiv \alpha_1 \mapsto \alpha_2 \mapsto \dots \mapsto \alpha_t
  \end{equation}
  where the instantaneous description $\alpha_t$ contains the terminal state symbol $q_0$.

  Applying the relations of Definition \ref{associated-semigroup} on $h\alpha_1 h$, considered an element of the associated semigroup, we use the relations corresponding to the basic moves in \ref{eq:bmoves} to have $hq_1\omega h = h \alpha_t h$. However the last three relations of $\Gamma(T)$ allow us to further reduce this to $q$, proving the necessary case of the theorem.

  Conversely, suppose that the element spelled $hq_1\omega h$ is equal to $q$ in the semigroup $\Gamma(T)$. By our discussion introducing elementary operations, we have that there exist words $\omega_2, \omega_3,\dots, \omega_t$ on the set
  \begin{equation*}
    \{ h,s_0,s_1,s_2, \dots, s_M, q_0, q_2, \dots, q_N \},
  \end{equation*}
such that there exist elementary operation as below
  \begin{equation*}
    hq_1 \omega h \equiv \omega_1 \rightarrow \omega_2 \rightarrow \dots \rightarrow \omega_t \equiv h q_0 h \rightarrow q.
  \end{equation*}

  Lemma \ref{lem:gamma-prop}(1) tells us that each $\omega_i$ is $h$-special and so for all $i$ we can write $\omega_i \equiv h \alpha_i h$ where $\alpha_i$ is an instantaneous description. If we can show for each $i$ that $\alpha_i \mapsto \alpha_{i+1}$ is a basic move in $T$, then $T$ computes $\omega$ and we are done.

  However this is not automatic, like the key step in the proof of the necessary condition was; this is for the same reason that we required \emph{either} $\alpha \mapsto \beta$ or $\beta \mapsto \alpha$ in Lemma \ref{lem:gamma-prop} (2); basic moves are not an symmetric relation while equality in the semigroup is. Thus we need to explicitly prove that $\alpha_i \mapsto \alpha_{i+1}$ for each $1 \geq i <t$ and \emph{not} $\alpha_{i+1} \mapsto \alpha_i$, which is the necessary alternative by Lemma \ref{lem:gamma-prop} (2). We will proceed by induction on $t \geq 2$.

  For $t = 2$ we have that the following is an elementary semigroup operation
  \begin{equation*}
    hq_1 \omega h  \rightarrow  h q_0 h,
  \end{equation*}
  and by Lemma \ref{lem:gamma-prop} (2), either $q_1 \omega \mapsto q_0$ or $q_0 \mapsto q_1 \omega$. However, the latter is not possible since $q_0$ is the terminal description and so the basic move must be the former, which is what we desire for the base case $t = 2$.

  In fact the argument for the base case shows that for all $t$, if
  \begin{equation*}
    hq_1 \omega h \equiv h\alpha_1h \rightarrow h\alpha_2 h \rightarrow \dots \rightarrow h\alpha_t h \equiv h q_0 h \rightarrow q,
  \end{equation*}
  Then $\alpha_{t-1} \mapsto \alpha_t$ must always be a basic move. Thus in the chain of basic moves between adjacent elements $\alpha_1, \alpha_2, \dots, \alpha_{t-1}, \alpha_t$, the last arrow points to the ``right''. If not all arrows point to the right, there is some $i$ such that $\alpha_i \mapsto \alpha_{i+1}$ \emph{and} $\alpha_i \mapsto \alpha_{i-1}$. We must have then that $\alpha_{i-1}$ and $\alpha_{i+1}$ are the same instantaneous description since Turing machines are not ambiguous. It follows that the words $\omega_{i-1}$ and $\omega_{i+1}$ have the same spelling and two words, $\omega_i and \omega_{i+1}$ can be deleted from the chain of semigroup elementary operations, preserving the rest. Consequently our chain will have fewer than $t$ words, but we have assumed by the inductive step that the desired result is true in this case.
\end{proof}

\begin{lemma}
  Let $\Omega$ be the set of all positive words on a set of symbols $S$. Suppose $\Omega' \subseteq \Omega$. If $E \subseteq \Omega$ is recursive then $E \cap \Omega'$ is a recursive subset of $\Omega'$
\end{lemma}

\begin{theorem}[Markov-Post]
  \begin{enumerate}
  \item There is a finitely presented semigroup
    \begin{equation*}
      \gamma = (q,h,s_0,s_1,s_2, \dots, s_M, q_0,q_2, \dots, q_N \mid R)
    \end{equation*}
    that has an unsolvable word problem.
  \item There is no decision process which determines, for an arbitrary $h$-special word $h\alpha h$ whether $h\alpha h = q$ in $\gamma$.
  \end{enumerate}
\end{theorem}


\begin{proof}
  For any Turing machine $T$ with stopping state $q_0$, we may suppose that its alphabet is $A = \{s_0,s_1,s_2, \dots, s_M \}$ and its $q$-letters are $B = \{q_0,q_1,q_2, \dots q_N\}$. We let $\Omega$ be the set of positive words on $A$. Introducing the new symbols $q$ and $h$ of the associated semigroup, we let $\bar{\Omega}$ to be the set of positive letters on $A \cup B \cup \{ q,h \}$. We then make the following definitions for this Turing machine:
  \begin{align*}
    E &:= e(T) = \{\omega \in \Omega \mid T \text{ computes } \omega\}; \\
    \bar{E} &:= \{\bar{\omega} \in \bar{\Omega} \mid  \bar{\omega} = q \text{ in } \Gamma(T) \}; \\
    \Omega_1 &:= \{hq_1\omega h \mid \omega \in \Omega\}; \\
    E_1 &:= \{hq_1\omega h \mid \omega \in E \}.
  \end{align*}

  Our aim is to show that there exists some $T^*$ such that the consequent $\bar{E}$ is not recursive in $\bar{\Omega}$. If we can find this, then the let $\gamma$ be the associated semigroup $\Gamma(T^*)$. This semigroup then has an unsolvable word problem by definition (for words evaluating to $q$ in the semigroup).

  For this, we note that by Lemma \ref{lem:semi-turing}, all elements in $\bar{E}$ of the form $hq_1\omega h$ have that $\omega \in E$ and the converse is true as well. Thus we have that
  \begin{equation*}
    E_1 = \bar{E} \cap \Omega_1
  \end{equation*}

  If $\bar{E}$ is recursive, then we claim that $E_1$ is a recursive as a subset of $\Omega_1$. Note that $\bar{E} \cap \Omega_1$ is recursively enumerable; so there is a Turing machine $T$ such that $e(T) = \bar{E}$. Now we can run a machine $T'$ that strips the $hq_1 \_ h$ from elements of $\Omega$ and then runs $T$ on the resultant word and $e(T') = E_1$. On the other hand, we note that $\Omega_1 - E_1 = \Omega_1 \cap (\bar{\Omega} - \bar{E})$. By hypothesis we know there exists a Turing machine that enumerates $\bar{\Omega} - \bar{E}$ and as before we can create a Turing machine that enumerates $\Omega_1 - E_1$.
  
  One can also see that $E_1$ is $E$ identified inside $\Omega_1 \subseteq \bar{\Omega}$, we have that it is recursive in $\Omega_1$ exactly when $E$ is recursive in $\Omega$.

  Putting it together, we have that if $\bar{E}$ is recursive in $\bar{\Omega}$, then $E$ is recursive in $\Omega$. Thus if we can find a Turing machine that enumerates a set that is not recursive, we are done. We have however done this in already in Theorem \ref{thm:natural}. Let $T^*$ be the Turing machine of this theorem that has the stopping state of $q_0$. Then $\gamma = \Gamma(T^*)$ has an unsolvable word problem.

  For the second part, we suppose
  \begin{equation*}
    \bar{S} = \{h\text{-special words }h\alpha h \mid h\alpha h = q. \text{ in } \Gamma(T^*) \}
  \end{equation*}
  Then with similar reasoning that we used for the first part, we can argue that since $\bar{S} \cap \Omega_1 = E_1$, that $\bar{S}$ cannot be a recursive subset of $\bar{\Omega}$.
\end{proof}

\begin{corollary}
  \label{cor:markov-post}
  \begin{enumerate}
  \item There exists a finitely presented semigroup
    \begin{equation*}
      \Gamma = \{q,q_0, \dots, q_N,s_0, \dots, s_M \mid F_iq_iG_i = H_iq_{i_2}K_i,i \in I\}
    \end{equation*}
    that has an unsolvable word problem, where $F_i,G_i,H_i,K_i$ are possibly empty positive $s$-words and $q_{i_1}$ and $q_{i_2}$ are $q$-letters.
  \item There is nod decision process that whether $X q_{i_j} Y = q$ in $\Gamma$, where $X$ and $Y$ are positive possibly empty $s$-words and $q_i$ is an arbitrary $q$-letter.
  \end{enumerate}
\end{corollary}

\begin{proof}
  For the first part we can simply take $\gamma$ from the previous theorem and let the $s$-letters be indexed such that $h$ is included among them (i.e.\ we add an extra $s$-letter: $s_M = h$). The relations written in this manner then have the desired form (recall Definition \ref{associated-semigroup}).

  Let $\Omega_2$ be the set of positive words on the generators of $\Gamma$ from the first part and suppose that
  \begin{equation*}
    \bar{\Lambda} = \{Xq_{i_j}Y  \mid X,Y \text{ are positive }s\text{-words and } Xq_{i_j}Y = q \text{ in } \Gamma\}.
  \end{equation*}
  and 
  \begin{gather*}
    \bar{S_2} = \{s_M\alpha s_M \mid \alpha \equiv \sigma q_{i_j} \tau \text{ where } \sigma,\tau \text{ are positive words on } s_0,\dots,s_{M-1} \\
    \text{ and } s_M\alpha s_M = q \text{ in } \Gamma\}
  \end{gather*}
  $\bar{S_2}$ is exactly $\bar{S}$ from the Theorem with the $h$'s relabeled, so we know that it is not recursive. As in the proof of the theorem, since $\bar{S_2} = \bar{\Lambda} \cap \Omega_2$, we have that $\bar{\Lambda}$ is not recursive.
\end{proof}

\section{The Novikov-Boone-Britton Theorem}

Our aim is now to investigate the word problem for groups. We start by defining what we want to show. Here, we again appeal to Church's thesis to link our informal understanding of decision processes to formal mathematics of Turing machines.

\begin{definition}
  Let $G$ be a group that is finitely presented by $(x_1,\dots,x_n \mid \Delta)$ and let $\Omega$ be the set of words on its generators. Then we say that $G$ has a \emph{solvable word problem} if $\{\omega \in \Omega \mid \omega = 1 \text{ in G}\}$ is recursive and an \emph{unsolvable word problem} if it is not.
\end{definition}

There does indeed exist a group that has an unsolvable word problem. This was proved independently by P.S.\ Novikov in 1955, by W.W.\ Boone in 1957 and by J.L.\ Britton in 1958 \cite{rotman1999introduction}. The proof uses Boone's Lemma which is crucial to the proof of the Theorem was proved using combinatorial group theory was proved again by Britton using properties of \emph{HNN extensions}, is a group that naturally extends a group with a presentation. We will state it here but not prove it. Unlike our discussion regarding semigroups, we now need to consider words on a set of symbols that  may have negative exponents as well. For this we introduce some notation to simplify things.

If $X \equiv s_{\beta_1}^{e_1} \dots s_{\beta_m}^{e_m}$ is any word on $s$-letters, then we let $X^{\#}$ be the word spelled $s_{\beta_1}^{-e_1} \dots s_{\beta_m}^{-e_m}$.

To show that the word problem for groups is unsolvable, our strategy will be to define a unique group, \emph{Boone's Group} for each Turing machine, similar to the associated semigroup from past sections.  This group for the Turing machine $T^*$ from the Markov-Post Theorem is the group with an unsolvable word problem that we seek.

\begin{definition}
  For a Turing machine $T$, we define \emph{Boone's Group} $\mathcal{B}(T)$ as the group with generators:
  \begin{equation*}
    q,q_0, \dots, q_N, s_0, \dots, s_M, r_i, i \in I,x,t,k;
  \end{equation*}
  and relations of the following types:
  \begin{align*}
    xs_\beta &= s_\beta x^2, \\
    r_is_\beta &= s_\beta xr_ix, \\
    r_i^{-1}F_i^{\#}q_{i_1}G_ir_is_\beta &= H_i^{\#}q_{i_2}K_i, \\
    tr_i &= r_it, \\
    tx &= xt, \\
    kr_i &= r_ik, \\
    kx &= xk, \\
    k(q^{-1}tq) &= (q^{-1}tq)k.    
  \end{align*}
\end{definition}

We are now ready to present Boone's Lemma, with some definitions and notation to simplify the statement.

\begin{definition}
  We say that a word is \emph{special} if it is spelled $X^{\#}q_jY$ if $X,Y$ are positive $s$-words and $q_j \in \{q,q_0,\dots,q_N\}$.
\end{definition}

For, $s$-words $X,Y$ and $q_j$ as in the above definition define the following operation as well:
\begin{equation*}
  (Xq_jY)^* \equiv X^{\#}q_jY.
\end{equation*}

\begin{lemma}[Boone]
  Let $T$ be a Turing machine with stopping state $q_0$ and associated semigroup $\Gamma$, that has been rewritten as in Corollary \ref{cor:markov-post}. If $\Sigma$ is a special word then we have that
  \begin{equation*}
    k(\Sigma^{-1}t\Sigma) = (\Sigma^{-1}t \Sigma)k
  \end{equation*}
  in $\mathcal{B}(T)$ if and only if $\Sigma^* = q$ in $\Gamma$.
\end{lemma}

The sufficient case for Boone's Lemma can be shown with the properties of associated semigroups that we have presented in this project, however the necessary case is much more difficult to prove. The proofs known for the necessary case depends on the \emph{Fundamental Theorem of Combinatorial Group Theory} and on \emph{Britton's Lemma} which is a property of HNN extensions, that we haven't introduced here. 

\begin{theorem}[Novikov-Boone-Britton]
  \label{thm:finale}
  There exists a finitely presented group with an unsolvable word problem.
\end{theorem}

\begin{proof}
  Let $\mathcal{B}$ be the Boone group for the Turing machine $T^*$ that we found for the Markov-Post Theorem. By the second part of Corollary \ref{cor:markov-post} there is no decision process that tells us whether $\Sigma^* = q$ in $\Gamma(T^*)$. Then by Boone's Lemma, we can't find a Turing machine that tells us whether
  \begin{equation*}
    k(\Sigma^{-1}t\Sigma) = (\Sigma^{-1}t \Sigma)k
  \end{equation*}
  in $\mathcal{B}$ and consequently whether
  \begin{equation*}
    k(\Sigma^{-1}t\Sigma)k^{-1}\Sigma^{-1}t^{-1}\Sigma = 1.
  \end{equation*}
  in $\mathcal{B}$.
\end{proof}


%%% Local Variables:
%%% mode: latex
%%% TeX-master: "main"
%%% End:
